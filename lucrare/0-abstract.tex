\begin{abstractpage}

\begin{abstract}{romanian}

  Computerele moderne folosesc tehnici de optimizare precum \emph{executie
  out-of-order} si \emph{branch prediction}. \emph{Meltdown} si \emph{Spectre}
  sunt doua atacuri care exploateaza efectele secundare aparute la nivel
  microarhitectural in urma optimizarilor mentionate. Prin intermediul acestora
  un atacator poate citi date private din zone arbitrare din memorie, fara
  privilegii si fara a exploata niciun bug de natura software. \emph{Intel},
  \emph{AMD} si \emph{ARM} au fost fortate in urma divulgarii acestor atacuri
  sa isi schimbe designul procesoarelor in incercarea de a mitiga
  vulnerabilitatile la nivel hardware. In ciuda solutiilor implementate, la
  jumatatea anului 2022, \emph{Spectre} afecteaza in continure majoritatea
  computerelor din lumea intreaga si ramane un pericol pentru utilizatori si un
  subiect de mare interes pentru cercetatori. In aceasta lucrare vor fi
  prezentate particularitatile celor doua atacuri, si o implementare
  demonstrativa a unui atac de tip \emph{Spectre}.

\end{abstract}

\begin{abstract}{english}
  
  Modern computers are equiped with features such as \emph{out-of-order
  execution} and \emph{branch prediction}, which are used to reduce CPU idel
  time and improve performance. \emph{Meltdown} and \emph{Spectre} are to cyber
  attacks that exploit microarhitectural side-effects which apper as a result
  of such optimization techiniques being used. An attacker can read private
  data of the vicim at arbitrary locations in memory, without exploiting any
  software bug. \emph{Intel}, \emph{AMD} and \emph{ARM} were forced to redesign
  their CPUs in order to migiate the risks posed by \emph{Meltdown} and
  \emph{Spectre}. Despite deployed mitigations, in the second half of 2022,
  most computers in the world are vulnerable to variations of \emph{Spectre}
  attacks, billions of users begin at risk. This class of attacks remains a
  subject of great interest for researchers in the field of security. In this
  work, the technicalities and implications of both attacks will be covered.
  Moreover, a proof of concept for a \emph{Spectre} attack will be presented.

\end{abstract}

\end{abstractpage}
