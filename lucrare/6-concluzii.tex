\chapter{Concluzii și Direcții Viitoare}

În această lucrare am abordat subiectul \emph{Atacurilor Speculative}. Am
prezentat detalile atacului \emph{Meltdown} care permitea unui utilizator
neprivilegiat citirea oricărei zone din memoria fizică prin intermediul
Kernelului, soluția care a dus la mitigarea acestuia (\emph{KAISER}), cât
și observații personale legate de reproducere. Ulterior am prezentat clasa
de atacuri \emph{Spectre} care permite citirea zonelor de memorie 
accesibile unui proces victimă cu care un proces atacator are în comun
o zonă limitată de memorie. Prezint de asemenea și soluții care au încercat
să mitigheze cu grade variate de succes aceste vulnerabilități. În final,
prezint o implementare inter-proces, cu scop didactic a \emph{Spectre v1}.

Noua variantă a \emph{Spectre} (\emph{BHI}) publicată la începutul anului 2022
demonstrează ineficiența solutilor implementate până în prezent importriva
\emph{Spectre V2}. Astfel, \emph{Spectre} rămâne un subiect deschis de
cercetare. Se pot dezvolta în continuare noi variante ale atacului care pot fi
eficiente chiar și împotriva mitigarilor ce vor apărea în viitorul apropiat. Se
pot dezvolta noi mitigari care să afecteze performanța programelor într-un mod
minimal, mai puțin decât cele deja existente care reduc performanța
considerabil. Se pot evalua implicările atacului pe platforme mai puțin
documentate.

În viitor, îmi doresc să studiez mai în detaliu \emph{Spectre V2} și ulterior,
noua variantă \emph{Spectre BHI} apărută recent, pentru a înțelege mai bine
mecanismele din spate care fac posibile rezultatele obținute. Noile cunoștinte
dobândite, m-ar ajuta în contiuarea cercetării atacurilor speculative.
