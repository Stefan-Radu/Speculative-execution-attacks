\chapter{Introducere}

\section{Context}

\emph{Meltodwn} și \emph{Spectre} fac parte din clasa largă a Atacurilor de tip
\emph{Side-Channel}, care exploatează mai degrabă efecte secundare rezultate
din implementarea unui sistem, decât defecte în implementarea algoritmilor ce
rulează pe acel sistem. De-a lungul timpului au apărut numeroase atacuri de tip
\emph{Side-channel}:

\begin{itemize}
  \item Timing attacks. Aceste atacuri au ca scop compromiterea unui sistem prin
    intermediul analizei statistice a timpilor de execuție ai unui algoritm, pe   
    diverse seturi de date de intrare (eg. compromiterea unui sistem de criptare
    RSA la distanță \cite{timing_practical})
  \item Cache attacks. Aceste atacuri se bazează pe abilitatea unui atacator de
    a monitoriza accesările unor zone de memorie partajate de către o victimă,
    și deducerea unor concluzii din modul în care aceste accesări influențează
    cache-ul procesorului. În cazuri extreme s-a demonstrat că se pot divulga
    chei criptografice secrete prin intermediul acestor tehnici
    \cite{percival2005cache}.
  \item Data remanence attacks. Aceste atacuri implică accesul asupra unor date
    după o presupusă ștergere a acestora în prealabil. Un exemplu clar este 
    atacul de tip \emph{Cold Boot} \cite{cold_boot} în care un atacator cu
    acces fizic la o mașină poate citi întreaga memorie RAM după efectuarea
    unui resetări a computerului.
  \item Rowhammer are un loc special în clasa de atacuri de tip \emph{Side-Channel}. 
    Prin accesul repetat al unei zone de memorie s-a observat că încărcătură electrică
    poate afecta zonele adiacente, provocând scurgeri de informație. Pe baza acestei
    tehnici s-au putut construi atacuri de tip escalare de privilegii \cite{rowhammer}.
  \item Mai există și alte tipuri de atacuri care exploatează consumul energetic,
    câmpul electromagnetic generat de componentele electronice, sau chiar și sunetul
    generat de sistem.
\end{itemize}

\emph{Meltodwn} și \emph{Spectre} se folosesc de idei asemănătoare cu cele
menționate în \emph{Timing Attacks} și în \emph{Cache Attacks} pentru a crea un
canal de comunicare ascuns (\emph{covert channel}). Pe acest canal se transmit
informații accesate în mod malițitos prin intermediul unor hibe în
implementarea la nivel hardware a computerelor moderne. Se va descrie cum
exploatarea acestor defecte este posibilă prin intermediul execuției
speculative în cadrul procesorului. Execuția speculativă presupune execuția în
avans a instructiunuilor pentru a salva timpi morți și a îmbunătăți
performanța. Fluxul de execuție poate fi manipulat în așa fel încât speculativ
să se execute instrucțiuni care nu s-ar executa vreodată în cadrul fluxului
normal de execuție al programului. Pentru menținerea consistentei și
corectitudinii rezultatelor obținute în urma execuției algoritmului,
rezultatele instrucțiunilor executate speculativ în mod \textbf{eronat} sunt
omise, iar starea internă este resetată la una anterioară, dar corectă. În
momentul resetării, starea cache-ului nu este și ea resetată, iar acest fapt
poate fi exploatat pentru obținerea unor informații secrete, care ulterior vor
putea fi transmise prin intermediul unui canal ascuns (\emph{covert channel}),
menționat anterior.

\section{Motivația Personală}

Din fire, încerc mereu să aprofundez cât mai în amănunt subiectele care mă
interesează, pentru a le înțelege în profunzime. Ca urmare, am ajuns atras de
subdomeniul securității care presupune o înțelegere profundă a sistemelor
informatice. Dintre atacurile cibernetice, \emph{Spectre} și \emph{Meltdown}
mi-au stârnit interesul și curiozitatea pe deoparte prin rezultatele
remarcabile obținute și pe de altă parte prin nivelul ridicat de subtilitate al
vulnerabilităților exploatate.

\section{Scopul Lucrării}

Lucrarea de față este rezultatul studiului personal al materialelor originale
care expuneau aceste atacuri. Aceasta are drept scop explicarea clară, dar
succintă a \emph{Meltdown} și \emph{Spectre}, într-un mod accesibil cititorilor
interesați. Este de asemenea prezentată și implementarea personală, aferentă
a unuia dintre atacurile discutate.

\section{Contribuții personale}

În urma studiului acestor tipuri de atacuri am reușit să realizez o
implementare cu scop demonstrativ a \emph{Spectre-v1} într-un scenariu mult mai
realist față de varianta prezentată în lucrarea de cercetare
\cite{spectre2019}. Implementarea mea propune un scenariu în care un proces
neprivilegiat citește în mod neautorizat zone de memorie dintr-un proces
victimă, fără acces la spațiul acestuia de memorie, ci doar la o zona limitată
de memorie partajată. În scopul realizării acestui exemplu au fost utilizate
tehnicile și noțiunile ce vor fi descrise pe parcursul acestei lucrări. Sursele sunt
publice și pot fi vizualizate la următoarele adrese:
\href{https://gist.github.com/Stefan-Radu/ca918598c9cce84429f566e020d93d15}{victima}
\footnote[1]{accesibil la: https://gist.github.com/Stefan-Radu/ca918598c9cce84429f566e020d93d15},
\href{https://gist.github.com/Stefan-Radu/29732c53a7d552fdec06fb46a801bd51}{atacator}
\footnote[2]{accesibil la: https://gist.github.com/Stefan-Radu/29732c53a7d552fdec06fb46a801bd51}.

\section{Structura Lucrării}

Lucrarea este împărțită în următoarele capitole:

\begin{enumerate}
  \item Introducere - se prezintă o viziune de ansamblu asupra atacurilor ce
    urmează să fie prezentate, un mic istoric al tehniclor, și motivația
    personală pentru realizarea acestei lucrări.
  \item Preliminarii - se prezintă noțiuni de \emph{Sisteme de Operare},
    \emph{Arhitectura Sistemelor de Calcul} și \emph{Securitate} relevante
    înțelegerii atacurilor discutate.
  \item Atacul Meltdown - se discută detalii de funcționalitate, metode de
    mitigare și detalii de reproducere a atacului Meltdown.
  \item Atacuri Spectre - se ilustrează deosebirile față de Meltdown, precum și
    detalii specifice de funcționalitate pentru cele două variante principale
    ale Spectre. Sunt prezentate metode de mitigare și starea actuală a
    vulnerabilităților.
  \item POC - Spectre-v1 - se prezintă o implementare demonstrativă a
    \emph{Spectre v1}
  \item Concluzii - se sumarizează cele discutate pe parcurs și se menționează
    direcțiile viitoare
\end{enumerate}
